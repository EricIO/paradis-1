\documentclass[12pt]{hitec}

\usepackage{fancyvrb}
%% \usepackage{hyperref}

\title{Assigments week 9}
\author{Joe Armstrong}

\begin{document}

\maketitle

\tableofcontents

\section{Introduction}

These problems are for lectures F12 and F13, held in week 9 of 2014.
The problems are in two categories:

\begin{itemize}
\item {\bf normal} solve these to get {\sl Godk\"{a}nd}.
\item {\bf advanced} solve these to get {\sl V\"{a}l Godk\"{a}nd}.
\end{itemize}

The sections and subsections in this paper are marked appropriately.

\section{Parallel Maps}

\verb+smap(F, L)+ (sequential map) maps a function \verb+F+ over a
list of items \verb+L+ It is defined as follows:

\begin{Verbatim}
smap(_, [])    -> []
smap(F, [H|T]) -> [F(H) | smap(F, T)].
\end{Verbatim}

In these exercises we will write a number of parallel mapping functions.

\subsection{normal: pmap\_max(F, L, Max)}

Write a function \verb+pmap_max(F, L, Max)+ that produces the same return
value as \verb+smap(F, L)+ by evaluating up to \verb+Max+ elements of the list
in parallel. Assume there are no exceptions raised when evaluating \verb+F+.

\subsection{normal: pmap\_any(F, L)}

Write a function \verb+pmap_any(F, L)+ that maps \verb+F+ over the
list \verb+L+ in parallel but where the items in the returned list
occur in the order they were computed.

Assume that:

\begin{Verbatim}
fib(1) -> 1;
fib(2) -> 1; 
fib(N) -> fib(N-1) + fib(N-2).
\end{Verbatim}

Then:

\begin{Verbatim}
> pmap_any(fun fib/1, [35,2,20]).
[1,6765,9227465]
\end{Verbatim}

Why is this? \verb+fib(2)+ is computed quickly so is first in the
list.  \verb+fib(35)+ comes last since it takes a long time to
compute. If we compute everything in parallel we should get the answer
to \verb+fib(2)+ first.

\subsection{normal: pmap\_any\_tagged(F, L)}

This is similar to the last exercise, with the addition that the returned
list should contain a copy of the input item that was processed. For example:

\begin{Verbatim}
> pmap_any_tagged(fun fib/1, [35,2,20]).
[{1,1},{20,6765},{35,9227465}]
\end{Verbatim}

\subsection{normal: pmap\_any\_tagged\_max\_time(F, L, MaxTime)}

This is similar to the last exercise, but with a maximum time 
(\verb+MaxTime+) by which all
parallel computations must terminate. \verb+MaxTime+ is a time in milliseconds.
All on-going computations what have not terminated within time
\verb+MaxTime+ must be killed. This means  \verb+pmap_any_tagged_max_times+
will always return within \verb+MaxTime+ milliseconds.

For example:

\begin{Verbatim}
> pmap_any_tagged_maxtime(fun fib/1, [35,2,456,20],2000).
[{1,1},{20,6765},{35,9227465}]
\end{Verbatim}

Note that the computation of \verb+fib(456)+ takes a very long time,
and was aborted.

{\sl Aside: How long would it take to calculate fib(456) using the above program?
Try estimating this. Can you think of a much faster way to compute fib(N)?}

\subsection{advanced: pmap\_timeout(F, L, Timeout, Max)}

This map function returns the elements in the list in the same order
as the original list. The arguments should be computed in parallel. If
a computation fails, the return value in the list should be the atom
\verb+error+.

If the the computation of an individual element takes more that
\verb+Timeout+ milliseconds then the computation should be aborted and
the return value in the list should be the atom \verb+timeout+.

For example:

\begin{Verbatim}
> pmap_timeout(fun fib/1, [10,20,2000,abc], 2000, 3).
[55,6765,timeout,error]
\end{Verbatim}

\subsection{advanced: pmap\_maxtime(F, L, MaxTime, Max)}

This is similar to the last exercise, but in this case \verb+MaxTime+ is to
be interpreted as the {\sl Total} time for all computations to take.

\section{Components}

\subsection{normal: Simple Middle Man}

Write a middle man to convert between metric and US systems of
measurement.

For example if we send a term like \verb+{km, 3}+ to the 
input port, it should send \verb+{miles,1.864}+ to the  output port.

See the lecture notes for a description of the middle man component.

\subsection{advanced: Components}

Implement the tracker and the tracker-client protocol described
in the lecture notes for lecture F13.

Note: This problem has a rather vague specification, so you will have to
make a number of design decisions to implement this. Document the
design decisions you had to make and motivate your choices.
 
\end{document}

