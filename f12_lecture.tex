\documentclass[10pt]{article}

\usepackage{fancyvrb}
%% \usepackage{hyperref}

\title{Parallel and Distributed Programs\\
Lecture F12\\
24 February, 2014
}
\author{Joe Armstrong}

\fvset{frame=single,numbers=left,numbersep=3pt}

\begin{document}

\maketitle

\tableofcontents

\section{Introduction}
In this lecture we'll look at some some ways of writing {\sl
  potentially parallel} programs.

In Erlang we write concurrent programs. The algorithms in the lecture
will run sequentially if run on a single core machine, but will run
in parallel if run on a multi-core computer. That's why they are
called ``potentially parallel.''


\section{Parallel Processing Functions}

To make a parallel program either the programmer must say what is to be
done in parallel, or the system must discover this automatically.

Automatic parallelisation of programs has been studied for many years.
It is very good for certain classes of problem (solving Laplace's
equation on a grid) and very bad for other classes of problem.

\subsection{ParBegin .. ParEnd}

Fro the early days of programming constructs have been added to
languages to allow programmers to express parallel computations:

\begin{verbatim}
ParBegin
   x;
   y;
   z;
ParEnd
\end{verbatim}

\verb+Fork+ ... \verb+Join+ is also used.

Both \verb+ParBegin+ .. \verb+ParEnd+ and \verb+Fork+ ... \verb+Join+ can be 
implemented using the lower level primitives \verb+promise+ and \verb+yield+

\subsection{Promises}

A ``{\sl Promise}'' is a commitment to deliver a value in the future.

\begin{verbatim}
    P 
    |
    |  promise(F)  
T1  +------>-------+
    |              |
    |              | 
    |              |
    +------<-------+
    |
    |
T2  +
    |
\end{verbatim}

At some time \verb+T1+ a process \verb+P+ wants to compute \verb+F()+ so it calls
\verb+promise(F)+ which returns a tag \verb+Tag+. \verb+P+ continues,
and \verb+F()+ is computed in parallel. At some time later (\verb+T2+) when 
\verb+P+ wants the return value it calls \verb+yield(Tag)+. \verb+yield+ blocks until
the result of the computation is available.

\VerbatimInput{promises.erl}

\verb+promises:test1()+ blocks until the result is available:

\begin{Verbatim}
> promises:test1().
.... yawn .. 
...  shell is blocked ...
... I can't do anything ...
165580141
\end{Verbatim}

Using a promise we can do {\sl useful things} \textregistered{} while
the promised computation is being performed:

\begin{Verbatim}
2> Tag = promises:test2().
Ref<0.0.0.104>
3> promises:query(Tag).
not_finished
4> 213813813861736*128763816381638163.
27531482667950334890792863030968
5> promises:query(Tag).               
not_finished
6> 19379127391273971239719379173979372193137*18763861238183638.
363627257286948472579176034410575587459129627874409292406
7> promises:query(Tag).                                        
{finished,165580141}
\end{Verbatim}

{\sl Note: If a command fails in the shell, the promise will not
be honoured, the results will be send back to the old shell which has died!}

\subsection{Broken Promises}

What happens if \verb+P+ dies {\sl Before} the worker has computed
its response? Should the worker be killed?

What happens if the worker dies while computing F. Who should be told?
Who should be killed?

Distributed Computing is the art of repairing broken promises.

\subsection{Maps}

Here is a simple implement of \verb+smap+ (Sequential map) and
\verb+pmap+ (Parallel map).

\VerbatimInput{maps.erl}

In the non-failure case \verb+smap+ and \verb+pmap1+ behave
identically.

\begin{Verbatim}
1> Double = fun(X) -> 2*X end.
#Fun<erl_eval.6.71889879>
2> maps:smap(Double,[1,2,3,4]).
[2,4,6,8]
3> maps:pmap(Double,[1,2,3,4]). 
[2,4,6,8]
\end{Verbatim}

Now let's try and double the list \verb+[1,2,3,a,3,4,b]+. In the
failure case \verb+smap+ will fail on the first error (ie when
computing \verb+2*a+) but \verb+pmap+ will fails for every element in
the list.

\begin{Verbatim}
4> maps:smap(Double,[1,2,a,3,4,b]).
** exception error: an error occurred when evaluating an 
   arithmetic expression
     in operator  */2
        called as 2 * a
     in call from maps:'-smap/2-lc$^0/1-0-'/2 (maps.erl, line 5)
     in call from maps:'-smap/2-lc$^0/1-0-'/2 (maps.erl, line 5)
5> maps:pmap(Double,[1,2,a,3,4,b]).
[2,4,
 {'EXIT',{badarith,[{erlang,'*',[2,a],[]},
                    {maps,'-pmap/2-fun-0-',3,
                          [{file,"maps.erl"},{line,10}]}]}},
 6,8,
 {'EXIT',{badarith,[{erlang,'*',[2,b],[]},
                    {maps,'-pmap/2-fun-0-',3,
                          [{file,"maps.erl"},{line,10}]}]}}]

\end{Verbatim}

\subsection{Broken Maps}

The idea of \verb+pmap+ seems simple {\sl but it is not}

\begin{description}
\item One dies, kill all\\ 
Assume one of the computations of
\verb+F(I)+ fails. We could kill all the other computations. But
what happens if we don't really care if the computation failed? Many
of the earlier computation might have suceeded so why drop the
results. The computations might have taken a long time, shame to
ignore the result.

\item Limit the concurrency\\
\verb+pmap+ will be {\sl too parallel} is we have a list of 1M processes
and the fucntion is {\sl Double} then pmaping would be wrong.
\end{description}



{\sl Discussion:} \verb+map+ and \verb+pmap+ have identical behaviour
when there are no errors. In \verb+pmap+ we fail immediately on the
first error. In \verb+pmap+ we cannot easily fail on the first error.
(We could link all the parallel processes togther, and let them all
die if one dies)

\section{Components}

\subsection{Server}

A server has state. When it receives requests from a client,
performes a computation which might update the state and then sends
a message back to the client.

(This is the server in the so call Client-Server Model)

\subsection{Publisher}

A publisher has a list of channels. \verb+{Channel, [Pid1, Pid2, ...]}+
The publisher accepts three input messages.

\begin{description}

\item \verb+{subscribe,Channel,Pid}+\\
  Adds \verb+Pid+ to the list
  of pids associated with \verb+ChannelName+. Creates a new channel if
  a channel does not exist.

\item \verb+{unsubscript_all,Pid}+\\
removes all subscriptions that Pid has.

\item \verb+{unsubscript_all,Channel, Pid}+\\
Remove the \verb+Pid+ to \verb+Channel+.
  
\item \verb+{publish,Channel,Msg}+\\ 
Sends Msg to all the subscribers associated with
\verb+Channel+. It is not an error to send a message to a channel with
zero subscribers.

\end{description}

Publisher has no errors.

A publisher is the PUB in the so called PUB-SUB model.

Note: PUB-SUB is an enormously important model:
Twitter, Facebook,  Mailing Lists, IRC Newsletters are all varients on PUB-SUB

Actually PUB-SUB is a confusing term. It could mean the
names of two fdffierent processes. A Publisher Process that talks to a Subscriber
Process. Or, it could mean the name of two different message sent to a publisher
process. 

\subsection{Dealer}
A dealer has \verb+N+ output ports. 
Messages sent to the input ports are round-robbined to
the output ports.

\begin{description}
\item \verb+{add_output,Pid}+\\
add Pid to the list of output ports.
\item \verb+{send,Message}+\\
  Send \verb+Message+ to one of the output ports.
  
  {\sl errors: There are no output ports}
\end{description}

\subsection{Router}

A router has an internal routing table which is a list of pairs of the form.
\verb+[{Tag,OutPort}]+. Message sent to the input port must have
a tag, these messages are sent to the appropriate output port.


\begin{description}
\item \verb+{add_route,Tag,Pid}+\\
  Adds a routing table entry \verb+{Tag, Pid}+ to the
  routing table. 

  {\sl errors: Tag already used}. 
  
\item \verb+{unroute,Tag}+\\
  Removes a routing table entry with \verb+Tag+ from the routing table.
  
\item \verb+{send,Tag,Msg}+\\
  Sends \verb+Msg+ to the process assocated with \verb+Tag+.
  
  {\sl errors: No process associated with Tag}
  
\end{description}
\section{Messaging Format}

All Messages are of the form: \verb+{do,Cmd,Epid}+
If the command works it does what it is supposed to do. If it fails
it sends a message to \verb+Epid+

\section{Router Code}

\VerbatimInput{router.erl}

\section{Publisher Code}

Here's a simple PUB-SUB publisher process. There is no error
handling.

\VerbatimInput{pubsub0.erl}

\end{document}


