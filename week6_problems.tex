\documentclass[12pt]{hitec}

\usepackage{fancyvrb}
%% \usepackage{hyperref}

\title{Assigments week 6}
\author{Joe Armstrong}

\begin{document}

\maketitle

\tableofcontents

\section{Introduction}
These problems are for lectures F6 and F7, held in week 6 of 2014.

Additional material can be found in Chapters 11 to 14 of the course book
{\sl Programming Erlang, 2'nd edition}.

\section{Problems}

This week I've put the problems into three categories:

\begin{itemize}
\item {\bf normal} solve these to get {\sl Godk\"{a}nd}.
\item {\bf advanced} solve these to get {\sl V\"{a}l Godk\"{a}nd}.
\item {\bf optional} solve these to get a deeper understanding of the problem. You don't
have to hand in your solutions to these, but solving them, or attempting them will
deepen your understanding of Erlang.
\end{itemize}

The sections and subsections in this paper are marked appropriately.

\section{Simple Processes}


Write the following routines:

\subsection{normal: \texttt{double:start()}}

\verb+double:start()+ creates a registered process called double.
If you send it a message containing an integer it should
double the integer and print the result.

If it is sent a non-integer it should print an error message
and crash.

Start the double process and test that it works correctly.

\subsection{normal: \texttt{monitor:start()}}

\verb+monitor:start()+ should create a processes that starts and
then monitors the double process that you made in the last section.
 
If the monitor process detects that the double process has crashed
it should print an error message and start a new double process.

Start the monitor process. Send the double process something that causes it
to crash. Then check that a new double process has been created and that
the system has recovered from the error.

\section{normal: Reliable Client-Server}

Here we assume that the server might crash. If a client sends a message to
and does not get a reply (this is detected by a timeout) the client
waits a random time and tries again. It tries this strategy several times
and gives up if the cannot get a reply from the server after 10 tries. 

\begin{itemize}

\item Create a registered server called double.

\item If you send it an integer it doubles it and sends back the reply.

\item It crashes if you send it an atom.

\item Make a process that sleeps for a random time and sends a message
  to the double server that causes it to crash.

\item Make a monitor process that detects that the server has
  crashed. It restarts the server after a random delay.

\item Make a client function that sends a request to the server and
  times out if the request is not satisfied. We can assume the server
  has crashed. The client should wait a second and then try again.

\item  Abort the client if it has tried more than ten times.
\end{itemize}

Hints: \verb+crypto:rand_uniform(Lo, Hi) -> N+
Generates a random number \verb+N+, where \verb+ Lo =< N < Hi+.

A sleep function can be defined as follows:

\begin{Verbatim}
sleep(T) ->
    %% sleep for T milliseconds
    receive
      after T ->
        true
    end.
\end{Verbatim}



\section{advanced: Replicated Data}

We're going to create two registered processes. Call them master and replica.
The idea is that the master process will have a data store and that
the replica process will have a copy of the store.

Normally we will update and request data from the master process, but
if the master process crashes we'll use the replica process.

Write the following routines:

\subsection{advanced: \texttt{store:start()}}.

starts the master and replica processes.

\subsection{advanced: \texttt{store:store(Key, Value)} and \texttt{store:fetch(Key)}}

\verb+store(Key, Value)+ stores a Key-Value pair in the data store.  A
master copy is kept in the master process, and a replica in the
replica process. \verb+fetch(Key)+ retrieves the data from the store.

Both \verb+store+ and \verb+fetch+ are interfacing functions which
send and receive messages between the client and the master and
replica processes.

The store should function correctly if the master processes crashes.

\subsection*{Making the program work for arbitrary crashes}

Note: In general distributed data replication is very difficult to
achieve.  If both the master and replica can crash at arbitrary times,
and if arbitrary messages can be lost the problem is impossibly
difficult to solve\footnote{There are some theorems about this, 
this is why I've only made a simple error case where only the master crashes.}.

\section{Sending messages in a ring}

\subsection{normal: \texttt{ring:start(N,M)}}

\verb+ring:start(N,M)+ should create a ring of \verb+N+ processes,
then send an integer \verb+M+ times round the ring. The message should
start as the integer 0 and each process in the ring should increment
the integer by 1.  After the message has been round the ring \verb+M+
times its final value should be \verb+N*M+. 

\verb+ring:start(N,M)+ should return the value of the last message
(which should be \verb+N*M+) and make sure that all the processes in
the ring have terminated.

\subsection{advanced: \texttt{ring:time(N,M)}}

Time how long it takes to run \verb+ring:start(N,M)+\\
(Hint, use
\verb+timer:tc(Mod, Func, Args)+\footnote{Consult the documentation for
  the timer module.}).

Measure the time it takes to send a message round the ring for
reasonably large values of \verb+N+ and \verb+M+. The product
of \verb+N*M+ should be a few million. Try this for tens of thousands to
millions of processes. Experiment with the \verb+erl -smp disable+.

\section{optional: A Mail Server}

Write a simple mail server. The server should respond to 4 messages:

\begin{description}

\item \verb+{message,From,Pwd,To,Subject,Data}+\\
  send a message to
  the server, all the arguments are binaries. When the message arrives
  at the server the server should check that the user \verb+From+ has
  password \verb+Pwd+ and that the user \verb+To+ exists. 

  The possible responses are:
  
  \begin{itemize}
  
  \item \verb+eNoSuchUser+ -- if the person \verb+To+ does not exist.
  
  \item \verb+eBadPassword+  -- if the person \verb+From+ does not have the
  password \verb+Pwd+.
  
\item \verb+{ok, MailId}+ -- if the mail was accepted. \verb+MailId+ is
  a unique identifier identifying this mail.
  The mail should be stored in a mailbox 
  associated with the user \verb+To+ together with a timestamp.
 

 \end{itemize}
  
 
\item \verb+{list_my_mail,User,Pwd}+\\
  This message is used to check for mail. 

  The possible responses are:
  
  \begin{itemize}
  
  \item \verb+eNoSuchUser+ -- if the person \verb+To+ does not exist.
  
  \item \verb+eBadPassword+  -- if the person \verb+From+ does not have the
  password \verb+Pwd+.
  
\item \verb+{ok, [{MailId,From,Header}]}+ -- returns a list of the mail
  waiting for the the user.
  \end{itemize}

 
\item \verb+{get_mail,MailId,Pwd}+\\
  This message is used to fetch a specific mail. \verb+Pwd+ is  
  the password of the user to whom the mail with Id \verb+mailId+ was sent.

  The possible responses are:
  
  \begin{itemize}
  
  \item \verb+error+ -- the mail Id does not exist or the password is incorrect.
  
\item \verb+{ok, [{MailId,From,Header,Content,Timestamp}]}+ -- returns the mail
  \end{itemize}

 
\item \verb+{delete_mail,MailId,Pwd}+\\
  This message is used to delete a specific mail. The arguments have the same
  meaning as for the \verb+get_mail+ command.

  The possible responses are:
  
  \begin{itemize}
  
  \item \verb+error+ -- the mail Id does not exist or the password is incorrect+
    
  \item \verb+{deleted, MailId}+ the mail was deleted.
    
  \end{itemize}

  

\end{description}

\subsection{A sample session with the mail server}

Assume Alice has password \verb+ecila+  and Bob \verb+blurb+

\begin{enumerate}

\item {\sl Alice sends a message to Bob}. She sends the message:

\begin{verbatim}
{mail,<<"alice">>,<<"ecila">>,<<"bob">>,
                 <<"hello">>,<<"This is ..">>)
\end{verbatim}

The mail server check that Alice's password is correct. It is correct so the
mail is added to Jim's mailbox. The server responds \verb+{ok,43}+. This is mail
number 43. The server adds a time stamp to the mail to record when it received
the mail.

\item {\sl Bob want to check his mail}. He sends the message:
 
\verb+{check_my_mail,<<"bob">>,<<"blurb">>>}+

To the server. Bob has supplied the correct password, so the server responds:

\verb+{mails, [..., {mail,43,<<"alice">>,<<"hello">>}, ...]}+

So Bob can see that he has got mail from Alice. 

\item Bob retrieves the mail by sending the message:

\verb+{get_mail,43,<<"blurb">>}+ 

to the server. He wants to read mail 43 
which the server knows is in Bob's mailbox. To do this Bob supplies his password
(\verb+blurb+). The password is correct so the mail server replies:

\begin{verbatim}
{msg,43,<<"alice">>,
        <<"hello">>,<<"This is ...">>,
        <<"2013-02-12 12:12:34">>}
\end{verbatim}



Bob reads the mail.

\item Bob deletes the mail by sending the message:

\verb+{delete_mail,43,<<"blurb">>}+ to the server. The server responds
\verb+{deleted,43}+

\end{enumerate}



\end{document}
