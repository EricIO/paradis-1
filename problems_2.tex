\documentclass[12pt]{hitec}

\usepackage{fancyvrb}
%% \usepackage{hyperref}

\title{Assigments week 5}
\author{Joe Armstrong}

\begin{document}

\maketitle

\tableofcontents

\section{Introduction}
These problems are for lectures E2 and E3.

Additional material can be found in Chapters 7..10 of the course book
{\sl Programming Erlang, 2'nd edition}.

\section{Problems}

The problems this week deal with binaries, the bit syntax, and types.

To get {\sl V\"{a}l Godk\"{a}nd} you solve all the problems marked
\verb+advanced+.

You should also write unit tests for all for functions and add type
declarations to all the exported functions. You should run the dialyzer
on your code, do not use \verb+-compile(export_all)+ when you run the dialyzer.

\section{LTV encoding}

Binaries and the bit syntax were introduced into Erlang specifically to
simplifying parsing data in communications protocols. 
One common scheme for representing data in
is the so called ``Tag-Length-Value'' encoding scheme.

Suppose we want to transmit strings, integers, and floating point numbers
between two programs on different machines. We can encode these as follows:

\begin{verbatim}
+-----------------------------------------+
| 1 | Len (4 bytes) | String of Len Bytes |
+-----------------------------------------+

+-------------------------------------------------+
| 2 | Len (4 bytes) | Integer encoded in Len bytes|
+-------------------------------------------------+

+------------------------------------------------+
| 3 | Len (4 bytes) | Float encoded in Len Bytes |
+------------------------------------------------+
\end{verbatim}

The first byte is the tag, 1 means ``string'', 2 means ``integer'' and
so on.  This is followed by a 4 byte length count encoded as a signed
big-endian integer. The remainder of the packet contains the encoding
of the string, integer or float.

Given this encoding write the following routines:

\subsection{\texttt{encode\_seq(L) -> Bin}}

\verb+encode_seq(L)+ takes an argument \verb+L+ which is a list of integers,
floats and strings and returns a binary \verb+Bin+ containing the
data in \verb+L+ as a sequence of  TLV encoded forms.

\subsection{\texttt{decode\_seq(Bin) -> L}}

Is the inverse of \verb+encode_seq+.


\subsection{advanced: \texttt{safe\_encode\_seq(L) -> Bin}}

Having written \verb+encode_seq+, you will realize there is a
problem. In Erlang integers are represented as bignums. If the
receiving program is C, then integers will have to be encoded as as
signed or unsigned, and we might want to distinguish between 32 and 64
bit integers. Also C does not understand bignums.

Add new tags to the encoding scheme to represent signed, unsigned, 32- and 64- bit
integers. Write \verb+safe_encode_seq+ to perform the encoding. This is the same
as \verb+encode_seq+ but it should use the extended tag scheme and it should
fail if called with invalid data (for example, if it is called with an integer
that will not fit into 64 bits).

 \subsection{advanced: \texttt{safe\_decode\_seq(L) -> Bin}}

This is the inverse of \verb+safe_encode_seq+.

\section{Packet manipulation}

We define a packet as a binary where the first four bytes represent a
4 byte length header \verb+N+ (encoded as a 32 bit big-endian integer)
followed by N bytes of data. Like this:

\begin{verbatim}
+--------------------------------+-------------------+
| N (4 bytes big-endian integer) | N bytes of Data   |
+--------------------------------+-------------------+
\end{verbatim}


The N bytes of data, can be binary data or an Erlang term \verb+T+
encoded with \verb+term_to_binary(T)+.


Write the following functions:

\subsection{\texttt{binary\_to\_packet(B)}}

Write a function \verb+binary_to_packet(B) -> P+ that converts a
binary \verb+B+ to a packet \verb+P+.

\subsection{\texttt{packet\_to\_binary(Packet)}}
Write the inverse function \verb+packet_to_binary(Packet)-> B+.


\subsection{\texttt{term\_to\_packet(Term)}}
Write the functions \verb+term_to_packet(Term) -> Packet+.

\subsection{\texttt{packet\_to\_term(Packet)}}

Write the inverse
function \verb+packet_to_term(P) -> Term+ where \verb+P+ is a packet.

\section{Detecting corrupt data}

When you have built a packet it might be a good idea to add a checksum
to the data, so that you can test it later to see if the packet has
been corrupted. The packet now looks like this:

\begin{verbatim}
+--------------+-------------------+
| N | Checksum | N bytes of Data   |
+--------------+-------------------+
\end{verbatim}

To compute a checksum you can call one of the functions
\verb+erlang:crc32+ or \verb+erlang:md5+.


\subsection{advanced: \texttt{term\_to\_packet\_with\_checksum(Term)}}
Write a safe version of \verb+term_to_packet+ which includes a checksum.

\subsection{advanced: \texttt{packet\_to\_term\_with\_checksum(Packet)}}
Write the inverse to \verb+term_to_packet_with_checksum(Term)+

{\sl Even more advance: is your code really safe -- is a checkum
  enough -- could a bad guy corrupt the data and cause a buffer
  overflow attack?}

\section{Markdown}

Markdown is a language for converting a simple text markup language
to HTML. The first exercise is to make a simple markdown processor.

Markdown is described at
\verb+http://daringfireball.net/projects/markdown/+.

Assume that the input to a markdown processor uses the following
conventions to delimit the different blocks in input:

\begin{enumerate}
\item Lines starting \verb+#+ denote HTML header (\verb+<h1>+) blocks.
\item Lines starting \verb+##+ denote HTML header (\verb+<h2>+) blocks.
\item Lines starting with any other characters are paragraphs.
\item Paragraphs are terminated by completely blank lines, or header blocks.
\end{enumerate}

\subsection{advanced: \texttt{markdown:expand\_binary(F)}}

Start by writing a function \verb+markdown:parse_binary(Bin)+ which
parses a binary and returns a list of markdown blocks in the
binary. For example:

\begin{Verbatim}[frame=single]
> markdown:parse_binary(<<"# hello
This is a paragraph
with two lines
#Another heading
and more data">>).
[{h1,<<"hello">>},
 {para,<<"This is a paragraph\nwith two lines\n"},
 {h1,<<"Another heading">>},
 {para,<<"and more data">>}]
\end{Verbatim}

\subsection{advanced: \texttt{markdown:parse\_to\_html(L)}}

Once we have a parse tree. We want to turn it into HTML, write a
function to expand the parser tree that you obtained in the previous
section, and convert it to HTML. For example:

\begin{Verbatim}[frame=single]
> markdown:parse_to_html([
  {h1,<<"hello">>},
  {para,<<"This is a paragraph\nwith two lines\n"},
  {h1,<<"Another heading">>}]).
  <<"<h1>Hello</h1><p>This is a ...">>.
\end{Verbatim}

\subsection{advanced: \texttt{markdown:expand\_file(F)}}

The function \verb+markdown:expand_file(F)+ should read the file
\verb+F+ using\\
\verb+file:read_file(F)+ the result is be a tuple
\verb+{ok, B}+ where \verb+B+ is a binary containing the content of
the file \verb+F+. Parse the resultant binary using
\verb+markdown:parse_binary(B)+ and turn the parse tree into HTML using
\verb+markdown:parse_to_html+. Write the result in a file
\verb+File.html+. Test this in a browser for various inputs.

\section{Notes}

In writing \verb+markdown:parse_binary+ you'll probably notice that the
markdown documentation is rather vague about whether things like leading
and trailing newlines belong to the paragraphs and headings or not.

Assume the input to the parser is:

\begin{Verbatim}
# Header1
The start of paragraph one ...
\end{Verbatim}

Does the newline separating the header and the first line of the
paragraph belong to the header or to the first line of the paragraph or
to neither?

Depending upon the answer to this question there are three possible
Erlang  parse trees, for example:

\begin{itemize}
\item \verb+[{h1,<<"Header1">>},{para,<<"The start ...">>},...]+
\item \verb+[{h1,<<"Header1\n">>},{para,<<"The start ...">>},...]+
\item \verb+[{h1,<<"Header1">>},{para,<<"\nThe start ...">>},...]+
\end{itemize}

What about the white space following the \verb+#+ character,
should this be represented in the parse tree or not?

Once we have a parse tree of the form \verb+[{h1,...}, {para, ...}]+
is is easy to render this as HTML or \LaTeX{}, note that any white
space issues (the kind I mentioned earlier) do not seem to affect the
rendered HTML.

How white space is handled is crucial to accurately rendering markdown,
if we extend the program to handle preformmated text, we'll get
problems if this is problem is not correctly analyzed.

{\sl For fun, you might like to connect this code with last weeks
  exercises in rendering inlines, to make a more sophisticated markdown
  processor.}

\end{document}
